\section{Introduction}
\subsection{Motivation and Background}
Since the publication of the Fortran 2008 standard in 2010~\cite{iso2010information}, Fortran supports a \gls{spmd}
programming style that facilitates the creation of a fixed number of replicas of a compiled program, wherein each replica
executes asynchronously after creation.  Fortran refers to each replica as an image.  The primary mechanism for distributing and
communicating data between images involves defining \glspl{coarray}, entities that may be referenced or defined on one image by
statements executing on other images.  As such, a coarray defines a \gls{pgas} in which one image referencing or defining a
coarray on another image causes inter-image communication.

The seminal role that \glspl{coarray} played in the development of Fortran's intrinsic parallel programming model have made it
common to refer to all of modern Fortran's parallel programming features under the rubric of \gls{caf}.  To date, most
published \gls{caf} applications involve scenarios wherein the parallelization itself poses one of the chief
challenges and necessitates the custom development of parallel algorithms.  These include ordinary and partial differential
equation solvers in domains ranging from nuclear fusion~\cite{preissl2011multithreaded} and
weather~\cite{mozdzynski2015partitioned} to multidimensional fast Fourier transforms and multigrid numerical
methods~\cite{garain2015comparing}.  Much of the effort involved in
expressing parallel algorithms for these domains centers on
 designing and using various \gls{coarray} data structures.  In such settings, the moniker
\gls{caf} seems appropriate.

Less widely appreciated are the ways Fortran's intrinsic parallel programming model supports embarrassingly
parallel applications, wherein the division into independent sub-problems requires little coordination between
the sub-problems. To support such applications, a parallel programming
model might provide for explicit sub-problem disaggregation and independent sub-problem execution without any need for
\gls{pgas} data structures such as \glspl{coarray}.  The draft Fortran 2018 standard
(previously named ``Fortran 2015''\footnote{A Committee Draft is at https://bit.ly/fortran-2015-draft.}) offers several
features that enable a considerable amount of parallel computation, coordination, and communication
even without \glspl{coarray}.  A working definition of ``embarrassingly parallel'' Fortran might denote the class of
use cases for which parallel algorithmic needs are met by the non-\gls{coarray} parallel features, including
\begin{itemize}
   \item Forming teams of images that communicate only with each other by default,
   \item Image synchronization: a mechanism for ordering the execution of program segments in differing images,
   \item Collective subroutines: highly optimized implementations of common parallel programming patterns,
   \item Image enumeration: functions for identifying and counting team members,
   \item Global error termination.
\end{itemize}
We anticipate that a common use case will encompass an ensemble of simulations, each member of
which executes as a parallel computation in a separate team.  This paper presents such a use case for
\gls{wrf-hydro}, a terrestrial hydrological model developed at \gls{ncar}.

\subsection{Objectives}\label{sec:objectives}
The objectives of the current work are threefold:
\begin{enumerate}
  \item To contribute the first-ever compiler front-end support for Fortran 2018 teams.
  \item To contribute the parallel runtime library functionality required to support the new compiler capabilities.
  \item To study and address issues arising form integrating teams into an existing \gls{mpi} application, \gls{wrf-hydro},
\end{enumerate}
The compiler front-end described herein lives on the teams branch of the
\fnurl{Sourcery Institute \gls{gcc} fork}{https://github.com/sourceryinstitute/gcc}. The parallel runtime library lives
on the opencoarrays-teams branch of the \fnurl{OpenCoarrays \gls{abi}}{https://github.com/sourceryinstitute/opencoarrays}
\gls{abi}.  We are unaware of any compiler support for Fortran 2018 teams other than that developed for the current
project.

%In the remainder of this paper, Section~\ref{sec:methodology} describes the teams
%support in the \gls{gcc} fork and in OpenCoarrays; Section~\ref{sec:discussion} discusses the use of teams
%in \gls{wrf-hydro} and a required language extension; Section~\ref{sec:conclusions} summarizes our conclusions
%and future work plans.

\section{Methodology}\label{sec:methodology}
Fortran 2008 does not provide for an environment in which
a subset of the images can easily work on part of an application without affecting
images outside the subset.  Grouping the images of a program into nonoverlapping
teams allows for more efficient and independent execution of parts of a larger
problem.  Multiphysics codes such as climate and weather models may benefit from the
consequent reduction in off-node communication, particularly when an entire team of images
fits within a single compute node.

\subsection{Teams in Fortran 2018}\label{teams-in-fortran-2015}
A team is a set of images that can execute independently of other images outside the team.
At program launch, all images comprise a team designated by a language-defined \texttt{initial\_team} integer identifier.
Except for the initial team, every team has a parent team in a one-to-many parent/child hierarchy.
A program executes a \texttt{form team} statement to create a mapping for subsequent grouping of images into
new child teams.  Each new team has an integer identifier that Fortran produces as the result of invoking the
\texttt{team\_number()} intrinsic function.  Information about the team to which the current image belongs can be
determined by the compiler from the collective value of the team variables on the images of the team.
All images execute the \texttt{form team} statement as a collective operation.

During execution, each image has a current team, which is only changed by execution of \texttt{change team} and
\texttt{end team} statements. Executing \texttt{change team} moves the executing image form the current team to a team
specified by a derived-type~\texttt{team\_type} variable.  Subsequent execution of a corresponding \texttt{end team}
statement restores the current team back to that team to which it belonged immediately prior to execution of the most
recent \texttt{change team}.

The \texttt{form team} statement takes an \texttt{team\_number} argument uniquely identifying the team and
a \texttt{team\_type} argument encapsulating other team information in private components.
Successful execution of a \texttt{form team} statement assigns the team-variable (of type \texttt{team type}) on each
participating image a value that specifies the new team to which the image will belong.
The \texttt{change team} statement takes as argument a \texttt{team type} variable that represents the new team to be used as
current team. The execution of the \texttt{end team} statement restores the current team back
to that immediately prior to execution of the \texttt{change team} statement.
Figures~\ref{fig:team-number-test}-\ref{fig:get-communicator-test} demonstrate the use of teams in two
unit tests from the OpenCoarrays repository.

\begin{figure}
  \lstinputlisting{figures/tests/team-number.f90}
  \caption{A unit test for the team-number function.\label{fig:team-number-test}}
\end{figure}

\begin{figure*}
  \lstinputlisting{figures/tests/get-communicator.f90}
  \caption{A unit test for the get\_communicator function.\label{fig:get-communicator-test}}
\end{figure*}

\subsection{Teams in \gls{gcc} and OpenCoarrays}\label{subsec:teams-in-gcc}
From an MPI perspective, a Fortran team is comparable to an MPI communicator. The \texttt{form team} statement is implemented in OpenCoarrays
using \texttt{MPI\_Comm\_split} and passing the team id argument as \texttt{color} for \texttt{MPI\_Comm\_split}.
In case of success, the resulting communicator is stored into a list of available teams.
Every element of the list keeps track of the association of team identifier and a MPI communicator.
The elements of the list of available teams gets returned by the function and stored into the \texttt{team type} variable.

The list of available and used teams are both initialized to team 1 (equivalent to \texttt{MPI\_COMM\_WORLD}) at the beginning of the execution.
When the \texttt{change team} statement gets invoked, the element of the list of available teams stored into the \texttt{team\_type} variable
gets passed as argument to the correspondent OpenCoarrays function. The \texttt{current\_team} variable used inside OpenCoarrays for
representing the current communicator gets reassigned with the value contained into the element of the list passed as argument.
Finally, a new element is added to the list of used teams. This list of element is nothing but a list of pointers to the elements of the list
of available teams. The insertion operation is always performed at the beginning of the list in order to keep track of the teams hierarchy.
An execution of the \texttt{end team} statements is implemented by removing the first element of the list of used teams and reassigning the
\texttt{current team} to the new first element of the list of used teams.

Figure~\ref{fig:teams} depicts schematically an initial team of images (black arrows) executing over time (progressing
downward) and able to coordinate and communicate through a global mechanism (black horizontal line).  At the point of executing
\texttt{form team} and \texttt{change team} statements, the compiler inserts references to the OpenCoarrays \gls{abi} into the
executable program.  Those references cause invocations of \texttt{MPI\_Split}, which in turn creates the colored groupings that
correspond to teams in Fortran 2018.

\begin{figure}
\includegraphics[width=0.4\textwidth]{figures/teams}
\vspace{-24pt}
\caption{Schematic of a program executing over time
  (left axis) in 12 images (top) communicating globally (black horizontal bar) and later within subgroups (colored horizontal bars).  Horizontal lines represent the communication mechanisms (default=solid, optional=dashed).  Fortran concepts are on the left.  Underlying \gls{mpi} concepts are on the right.\label{fig:teams}}
\end{figure}
%

\begin{figure}
\includegraphics[width=0.4\textwidth]{figures/WRF-Hydro-caf-ens-model_chain.png}
\vspace{-7pt}
\caption{Caffeination of NCAR's WRF-Hydro model using Fortran 2018
  teams. Example components of  the National Water
  Model are shown. Different \gls{mpi} colors represent independent teams,
  each of which is one member of an ensemble of simulations.
  This approach supports introducting \gls{caf} incrementally
  into the WRF-Hydro Model Chain
  \label{fig:caffeinate-wrf-hydro}}
\end{figure}
%

The teams unit tests in
Figures~\ref{fig:team-number-test}--\ref{fig:get-communicator-test} use a block distribution of images,
dividing the initial team into three new teams, each with the same number of images except a number of teams
with one extra. The number of teams with one extra equals the remainder of integer division of the total
number images by the number of teams. Figure~\ref{fig:get-communicator-test} contains an assertion
utility that executes global error termination across all images if the logical expression passed as its
first argument is false.  The optional second argument in some
\texttt{assert} subroutine invocations in Figures~\ref{fig:team-number-test}--\ref{fig:get-communicator-test}
describes the checks performed for the teams features discussed in this paper.

\subsection{A language extension}
Line 2 in Figure~\ref{fig:get-communicator-test} imports a \texttt{get\_communicator()} function via Fortran's
use-association mechanism for accesing entities in Fortran modules: an \texttt{opencoarrays} module that
provides language extensions.  Lines 8 and 12 invoke this function
to provide the \gls{mpi} communicator to the test subroutine \texttt{mpi\_mpatches\_caf}.  Assertions in the
latter subroutine verify the expected correspondences between \gls{mpi} image and rank numbering.
The \gls{mpi}/\gls{caf} correspondence enables the newly caffeinated \gls{wrf-hydro}
to interoperate safely with the existing \gls{wrf-hydro} \gls{mpi} code.

\section{Discussion of Results}\label{sec:discussion}
\gls{wrf-hydro} is a community hydrologic model providing a parallel-computing
framework for coupling weather prediction (Figure \ref{fig:caffeinate-wrf-hydro}a), land surface
(Figure \ref{fig:caffeinate-wrf-hydro}b), and hydrologic routing to handle spatial water redistribution
via overland flow (Figure \ref{fig:caffeinate-wrf-hydro}c), subsurface (soil column) flow (Figure \ref{fig:caffeinate-wrf-hydro}c),
baseflow (deep groundwater, \ref{fig:caffeinate-wrf-hydro}d), and stream channel transport
(Figure \ref{fig:caffeinate-wrf-hydro}e)~\cite{gochisEtal2014}.

Developed to couple land hydrology to the atmosphere,
WRF-Hydro is most often run ``offline'' with one-way-coupled
forcing from upper-boundary (weather) conditions (Fig.~\ref{fig:caffeinate-wrf-hydro}).
The chief example is \gls{nwm} of \gls{noaa}~\cite{noaa2016}, a  special
configuration of WRF-Hydro providing operational, real-time analysis and forecasts of
hydrologic states over the contiguous U.S. (\ref{fig:caffeinate-wrf-hydro}).

Ensemble forecasting and ensemble data assimilation are
growing research topics for WRF-Hydro. Running ensembles under a
single executable program and job submission on HPC platforms as shown in
Fig. \ref{fig:caffeinate-wrf-hydro} can greatly reduce the
labor in designing workflows and can open up new
possibilities for improving data flows and optimizing ensemble
execution time.

\begin{figure}
\includegraphics[width=0.3\textwidth]{figures/init_timing_linear.png}
\vspace{-7pt}
\caption{Scaling of the channel-only initialization time to
  support Amdhal's law calculation of potential speedup of ensemble
  computation with shared initialization. \label{fig:wrf-hydro-init-scaling}}
\end{figure}

An optimization that we expect teams to facilitate will involve sharing initialization
work over all ensemble members using
the full image set before changing teams. Amortizing common
initialization work across all available computational resources
rather than repeating common initialization work in each
ensemble member using only that member's fraction of the resources might
save significant computational costs in runs for which
initialization occupies a large fraction of
the run time such as in production runs of short-term \gls{nwm}
forecasts. We are exploring
moving from deterministic
(single ensemble member) runs to ensemble runs for the stream
channel submodel in isolation. Fig.~\ref{fig:wrf-hydro-init-scaling}
presents the scaling behavior of channel-only runs.
Running 50 ensemble members concurrently in
teams occupying 20 cores each, the initialization speedup will be
approximately 20x. Currently, for the short-range forecasts,
initialization requires approximately 50\% of the run time on
NCAR's yellowstone computer and greater than that on \gls{noaa} WCOSS
system~\cite{yuetal2017}. These numbers are
similar for the channel-only model. In our proposed application for an ensemble size of 50, we estimate that
approximately half the runtime ($f=.5$) can be sped up by a factor of
about 20 ($S_f=20$), based on Fig.~\ref{fig:wrf-hydro-init-scaling}. From
these estimates, Amdhal\'s law yields a total speedup
of 1.9:
\begin{equation}
S = 1 / { f/S_f + (1-f) } = 1 / (.5/20 + (1-.5)) = 1.9
\end{equation}
To apply teams to \gls{wrf-hydro}, we converted the main program to a subroutine
invoked inside a loop over the ensemble members.  To put \gls{caf} in the driver seat,
we eliminated the \texttt{MPI\_Init} and \texttt{MPI\_FInalize} calls from the original
main program and instead pass a communicator to the converted main program.o  To facilitate
looping over the ensemble, we also eliminated the \texttt{stop} statement in the converted
main program.

\section{Conclusions and Future Work}\label{sec:conclusions}
We applied Fortran 2018 to ensemble simulation and forecasting with \gls{wrf-hydro}.
We implemented the first-ever compiler front-end and parallel runtime library support
for teams.  We developed a language extension that exposes \gls{caf}'s
underlying \gls{mpi} communicator for use in \gls{wrf-hydro}.  This approach facilitates
incremental introduction of \gls{caf}, i.e.,``caffeination,'' of the
pre-existing \gls{mpi} program.  We predict a speedup of 1.9 in ensemble execution when
spreading common initializations across all images prior to changing teams.  Such a speedup
would grealy impact \gls{wrf-hydro} operational use cases.

%%%
%%% Sample tables
%%%

%\begin{table}
%  \caption{Frequency of Special Characters}
%  \label{tab:freq}
%  \begin{tabular}{ccl}
%    \toprule
%    Non-English or Math&Frequency&Comments\\
%    \midrule
%    \O & 1 in 1,000& For Swedish names\\
%    $\pi$ & 1 in 5& Common in math\\
%    \$ & 4 in 5 & Used in business\\
%    $\Psi^2_1$ & 1 in 40,000& Unexplained usage\\
%  \bottomrule
%\end{tabular}
%\end{table}

%\begin{table*}
%  \caption{Some Typical Commands}
%  \label{tab:commands}
%  \begin{tabular}{ccl}
%    \toprule
%    Command &A Number & Comments\\
%    \midrule
%    \texttt{{\char'134}author} & 100& Author \\
%    \texttt{{\char'134}table}& 300 & For tables\\
%    \texttt{{\char'134}table*}& 400& For wider tables\\
%    \bottomrule
%  \end{tabular}
%\end{table*}
% end the environment with {table*}, NOTE not {table}!

%It is strongly recommended to use the package booktabs~\cite{Fear05}
%and follow its main principles of typography with respect to tables:
%\begin{enumerate}
%\item Never, ever use vertical rules.
%\item Never use double rules.
%\end{enumerate}
%It is also a good idea not to overuse horizontal rules.


%%%
%%% Sample figures
%%%

%\begin{figure}
%\includegraphics{fly}
%\caption{A sample black and white graphic.}
%\end{figure}

%\begin{figure}
%\includegraphics[height=1in, width=1in]{fly}
%\caption{A sample black and white graphic
%that has been resized with the \texttt{includegraphics} command.}
%\end{figure}

%\begin{figure*}
%\includegraphics{flies}
%\caption{A sample black and white graphic
%that needs to span two columns of text.}
%\end{figure*}
%
%\begin{figure}
%\includegraphics[height=1in, width=1in]{rosette}
%\caption{A sample black and white graphic that has
%been resized with the \texttt{includegraphics} command.}
%\end{figure}

%\end{document}  % This is where a 'short' article might terminate



%\appendix
%Appendix A
%\section{Teams unit tests}
% This next section command marks the start of
% Appendix B, and does not continue the present hierarchy

%\begin{figure*}
%\includegraphics[width=0.7\textwidth]{figures/hydro-map}
%\vspace{-36pt}
%\caption{Sample \gls{wrf-hydro} simulation domain.}
%\end{figure*}
%
\begin{acks}
The first author thanks the
\gls{ncar} Visitor Program for supporting a visit to \gls{ncar} during which
much of the work described herein was performed.

%  The work is
%  supported by the \grantsponsor{GSxxxxx}{National
%  Science Foundation China}{http://dx.doi.org/zz.yyyyy/xxxxx} under Grant
%  No.:~\grantnum{GSxxxxx}{yyyyyyy}

\end{acks}
