\documentclass[sigconf, authordraft]{acmart}

\usepackage{booktabs} % For formal tables


% Copyright
%\setcopyright{none}
%\setcopyright{acmcopyright}
%\setcopyright{acmlicensed}
\setcopyright{rightsretained}
%\setcopyright{usgov}
%\setcopyright{usgovmixed}
%\setcopyright{cagov}
%\setcopyright{cagovmixed}


% DOI
\acmDOI{xx.xxx/xxx_x}

% ISBN
\acmISBN{xxx-xxxx-xx-xxx/xx/xx}

%Conference
\acmConference[PAW17]{PGAS Applications Workshop}{November 2017}{Denver, Colorado USA}
\acmYear{2017}
\copyrightyear{2017}

%\acmPrice{15.00}

\acmSubmissionID{xxx-xxx-xx}

\begin{document}
\title{On the incremental caffeination of a terrestrial hydrological modeling framework}
%\titlenote{Produces the permission block, and
%  copyright information}
%\subtitle{Extended Abstract}
%\subtitlenote{The full version of the author's guide is available as
%  \texttt{acmart.pdf} document}

\author{Damian Rouson}
\orcid{0000-0002-2234-868X}
\affiliation{%
  \institution{Sourcery Institute}
  \streetaddress{2323 Broadway}
  \city{Oakland}
  \state{California}
  \postcode{94612}
}
\email{damian@sourceryinstitute.org}
\renewcommand{\shortauthors}{D. Rouson et al.}

\author{James McCreight}
\orcid{0000-000?-????-????}
\affiliation{%
  \institution{National Center for Atmospheric Research}
  \streetaddress{ 3450 Mitchell Ln.}
  \city{Boulder}
  \state{Colorado}
  \postcode{80301}
}
\email{jamesmcc@ucar.edu}

\author{Alessandro Fanfarillo}
\orcid{0000-0003-3487-7452}
\affiliation{%
  \institution{National Center for Atmospheric Research}
  \streetaddress{ 1850 Table Mesa Dr.}
  \city{Boulder}
  \state{Colorado}
  \postcode{80305}
}
\email{elfanfa@ucar.edu}


\begin{abstract}
We present a coarray Fortran (CAF) use case involving the initial steps in transitioning
a moderately large production code from using the Message Passing Interface (MPI)
to using mixed CAF/MPI.  In particular, we exploit recently developed support for
Fortran 2015 process teaming to demonstrate how to replace job scripts previously
used to launch ensemble runs of the WRF-Hydro hydrological model developed
at the National Center for Atmospheric Research.  For this purpose, we developed the
first implementation of compiler and parallel runtime library support for Fortran 2015
teams using a public fork of the GNU Compiler Collection (GCC) Fortran front end and
an open-source branch of the OpenCoarrays application binary interface.  We believe
the approach presented represents a common use case that will be representative of the
class of embarrassingly parallel applications, a class of problems that falls outside the
most common published demonstrations of CAF.  We further discuss the potential for
applying the same strategy to incrementally introduce CAF into an existing MPI code.
\end{abstract}

%
% The code below was generated by the tool at
% http://dl.acm.org/ccs.cfm
%
\begin{CCSXML}
<ccs2012>
<concept>
<concept_id>10011007.10011006.10011008.10011009.10010175</concept_id>
<concept_desc>Software and its engineering~Parallel programming languages</concept_desc>
<concept_significance>500</concept_significance>
</concept>
<concept>
<concept_id>10010405.10010432.10010437.10010438</concept_id>
<concept_desc>Applied computing~Environmental sciences</concept_desc>
<concept_significance>300</concept_significance>
</concept>
</ccs2012>
\end{CCSXML}

\ccsdesc[500]{Software and its engineering~Parallel programming languages}
\ccsdesc[300]{Applied computing~Environmental sciences}

\keywords{coarray Fortran, computational hydrology, parallel programming}

\maketitle

\section{Introduction}
In the beginning, there was FORTRAN.

%%%
%%% Sample tables
%%%

%\begin{table}
%  \caption{Frequency of Special Characters}
%  \label{tab:freq}
%  \begin{tabular}{ccl}
%    \toprule
%    Non-English or Math&Frequency&Comments\\
%    \midrule
%    \O & 1 in 1,000& For Swedish names\\
%    $\pi$ & 1 in 5& Common in math\\
%    \$ & 4 in 5 & Used in business\\
%    $\Psi^2_1$ & 1 in 40,000& Unexplained usage\\
%  \bottomrule
%\end{tabular}
%\end{table}

%\begin{table*}
%  \caption{Some Typical Commands}
%  \label{tab:commands}
%  \begin{tabular}{ccl}
%    \toprule
%    Command &A Number & Comments\\
%    \midrule
%    \texttt{{\char'134}author} & 100& Author \\
%    \texttt{{\char'134}table}& 300 & For tables\\
%    \texttt{{\char'134}table*}& 400& For wider tables\\
%    \bottomrule
%  \end{tabular}
%\end{table*}
% end the environment with {table*}, NOTE not {table}!

%It is strongly recommended to use the package booktabs~\cite{Fear05}
%and follow its main principles of typography with respect to tables:
%\begin{enumerate}
%\item Never, ever use vertical rules.
%\item Never use double rules.
%\end{enumerate}
%It is also a good idea not to overuse horizontal rules.


%%%
%%% Sample figures
%%%

%\begin{figure}
%\includegraphics{fly}
%\caption{A sample black and white graphic.}
%\end{figure}

%\begin{figure}
%\includegraphics[height=1in, width=1in]{fly}
%\caption{A sample black and white graphic
%that has been resized with the \texttt{includegraphics} command.}
%\end{figure}

%\begin{figure*}
%\includegraphics{flies}
%\caption{A sample black and white graphic
%that needs to span two columns of text.}
%\end{figure*}
%
%\begin{figure}
%\includegraphics[height=1in, width=1in]{rosette}
%\caption{A sample black and white graphic that has
%been resized with the \texttt{includegraphics} command.}
%\end{figure}

%\end{document}  % This is where a 'short' article might terminate



\appendix
%Appendix A
\section{Code listing}
% This next section command marks the start of
% Appendix B, and does not continue the present hierarchy
\section{Anything else?}

\begin{acks}
  The authors would like to thank the CISL and RAL Visitor Program

  The work is
  supported by the \grantsponsor{GSxxxxx}{National
  Science Foundation China}{http://dx.doi.org/zz.yyyyy/xxxxx} under Grant
  No.:~\grantnum{GSxxxxx}{yyyyyyy}

\end{acks}


\bibliographystyle{ACM-Reference-Format}
\bibliography{bibliography}

\end{document}
