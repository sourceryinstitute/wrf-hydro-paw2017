\documentclass[sigconf]{acmart}
%\documentclass[sigconf, authordraft]{acmart}

\usepackage{listings}
\usepackage{booktabs} % For formal tables

\usepackage{hyperref}
\newcommand\fnurl[2]{%
  \href{#2}{#1}\footnote{\url{#2}}%
}

\usepackage[acronym,nonumberlist]{glossaries}
\makeglossaries

\lstdefinelanguage{Fortran2015}{%
  language     = [08]Fortran,
  morekeywords = {
    co_sum,
    co_broadcast,
    co_min,
    co_max,
    co_reduce,
    event_type,
    event_query,
    event wait,
    event post,
    team_umber,
    team_type,
    form team,
    change team,
    end team,
  },
}

\lstset{ %
  basicstyle=\footnotesize,        % the size of the fonts that are used for the code
  breakatwhitespace=false,         % sets if automatic breaks should only happen at whitespace
  breaklines=true,                 % sets automatic line breaking
  captionpos=b,                    % sets the caption-position to bottom
  escapeinside={\%*}{*)},          % if you want to add LaTeX within your code
  frame=single,                    % adds a frame around the code
  keepspaces=true,                 % keeps spaces in text, useful for keeping indentation of code (possibly needs columns=flexible)
  language=Fortran2015,            % the language of the code
  morekeywords={procedure,iso_fortran_env},  % if you want to add more keywords to the set
  numbers=left,                    % where to put the line-numbers; possible values are (none, left, right)
  numbersep=5pt,                   % how far the line-numbers are from the code
  numberstyle=\tiny,               % the style that is used for the line-numbers
  showspaces=false,                % show spaces everywhere adding particular underscores; it overrides 'showstringspaces'
  showstringspaces=false,          % underline spaces within strings only
  showtabs=false,                  % show tabs within strings adding particular underscores
  keywordstyle=\color{brown}\ttfamily,
  commentstyle=\color{cyan}\ttfamily,
  stringstyle=\color{orange}\ttfamily,
 %title=\lstname                   % show the filename of files included with \lstinputlisting; also try caption instead of title
}


% Copyright
%\setcopyright{none}
%\setcopyright{acmcopyright}
%\setcopyright{acmlicensed}
\setcopyright{rightsretained}
%\setcopyright{usgov}
%\setcopyright{usgovmixed}
%\setcopyright{cagov}
%\setcopyright{cagovmixed}

\input glossary/glossary.tex
\input glossary/acronyms.tex

\begin{document}

\copyrightyear{2017}
\acmYear{2017}
\setcopyright{acmlicensed}
\acmConference[PAW17]{PAW17: Second Annual PGAS Applications Workshop}{November 12--17, 2017}{Denver, CO, USA}
%\acmBooktitle{P A W17: P A W17: Second Annual PGAS Applications Workshop, November 12--17, 2017, Denver, CO, USA}
\acmPrice{15.00}
\acmDOI{10.1145/3144779.3169110}
\acmISBN{978-1-4503-5123-2/17/11}

\title{Incremental caffeination of a terrestrial hydrological modeling framework using Fortran 2018 teams}
%\titlenote{Produces the permission block, and
%  copyright information}
\subtitle{Extended Abstract}
%\subtitlenote{The full version of the author's guide is available as
%  \texttt{acmart.pdf} document}


\author{Damian Rouson}
\orcid{0000-0002-2234-868X}
\affiliation{%
  \institution{Sourcery Institute}
  \streetaddress{2323 Broadway}
  \city{Oakland}
  \state{California}
  \postcode{94612}
}
\email{damian@sourceryinstitute.org}
\renewcommand{\shortauthors}{D. Rouson et al.}

\author{James L. McCreight}
\orcid{0000-0001-6018-425X}
\affiliation{%
  \institution{Natl. Ctr. for Atmospheric Research}
  \streetaddress{ 3450 Mitchell Ln.}
  \city{Boulder}
  \state{Colorado}
  \postcode{80301}
}
\email{jamesmcc@ucar.edu}

\author{Alessandro Fanfarillo}
\orcid{0000-0003-3487-7452}
\affiliation{%
  \institution{Natl. Ctr. for Atmospheric Research}
  \streetaddress{ 1850 Table Mesa Dr.}
  \city{Boulder}
  \state{Colorado}
  \postcode{80305}
}
\email{elfanfa@ucar.edu}

\begin{abstract}
We present Fortran 2018 teams (grouped processes) running a parallel ensemble of simulations
built from a pre-existing \gls{mpi} application.  A challenge arises
around the Fortran standard's eschewing any direct reference to lower-level communication
substrates, such as \gls{mpi}, leaving any interoperability between Fortran's
parallel programming model, \gls{caf}, and the supporting substrate to the quality of the
compiler implmentation.  Our approach introduces  \gls{caf} incrementally, a process we term
``caffeination.''  By letting \gls{caf} initiate execution and exposing
the underlying \gls{mpi} communicator to the original application
code, we create a one-to-one correspondence between \gls{mpi} group colors and Fortran
teams.  We apply our approach to the \gls{ncar}'s \gls{wrf-hydro}.
The newly caffeinated main program replaces batch job submission scripts and forms teams
that each execute one ensemble member.  To support this work, we developed the first
compiler front-end and parallel runtime library support for teams.  This paper describes
the required modifications to a public \gls{gcc} fork, an
OpenCoarrays~\cite{fanfarillo2014opencoarrays} \gls{abi} branch, and a \gls{wrf-hydro}
branch.
\end{abstract}

%
% The code below was generated by the tool at
% http://dl.acm.org/ccs.cfm
%
\begin{CCSXML}
<ccs2012>
<concept>
<concept_id>10011007.10011006.10011008.10011009.10010175</concept_id>
<concept_desc>Software and its engineering~Parallel programming languages</concept_desc>
<concept_significance>500</concept_significance>
</concept>
<concept>
<concept_id>10010405.10010432.10010437.10010438</concept_id>
<concept_desc>Applied computing~Environmental sciences</concept_desc>
<concept_significance>300</concept_significance>
</concept>
</ccs2012>
\end{CCSXML}

\ccsdesc[500]{Software and its engineering~Parallel programming languages}
\ccsdesc[300]{Applied computing~Environmental sciences}

\keywords{coarray Fortran, computational hydrology, parallel programming}

\maketitle

\section{Introduction}
In the beginning, there was FORTRAN.

%%%
%%% Sample tables
%%%

%\begin{table}
%  \caption{Frequency of Special Characters}
%  \label{tab:freq}
%  \begin{tabular}{ccl}
%    \toprule
%    Non-English or Math&Frequency&Comments\\
%    \midrule
%    \O & 1 in 1,000& For Swedish names\\
%    $\pi$ & 1 in 5& Common in math\\
%    \$ & 4 in 5 & Used in business\\
%    $\Psi^2_1$ & 1 in 40,000& Unexplained usage\\
%  \bottomrule
%\end{tabular}
%\end{table}

%\begin{table*}
%  \caption{Some Typical Commands}
%  \label{tab:commands}
%  \begin{tabular}{ccl}
%    \toprule
%    Command &A Number & Comments\\
%    \midrule
%    \texttt{{\char'134}author} & 100& Author \\
%    \texttt{{\char'134}table}& 300 & For tables\\
%    \texttt{{\char'134}table*}& 400& For wider tables\\
%    \bottomrule
%  \end{tabular}
%\end{table*}
% end the environment with {table*}, NOTE not {table}!

%It is strongly recommended to use the package booktabs~\cite{Fear05}
%and follow its main principles of typography with respect to tables:
%\begin{enumerate}
%\item Never, ever use vertical rules.
%\item Never use double rules.
%\end{enumerate}
%It is also a good idea not to overuse horizontal rules.


%%%
%%% Sample figures
%%%

%\begin{figure}
%\includegraphics{fly}
%\caption{A sample black and white graphic.}
%\end{figure}

%\begin{figure}
%\includegraphics[height=1in, width=1in]{fly}
%\caption{A sample black and white graphic
%that has been resized with the \texttt{includegraphics} command.}
%\end{figure}

%\begin{figure*}
%\includegraphics{flies}
%\caption{A sample black and white graphic
%that needs to span two columns of text.}
%\end{figure*}
%
%\begin{figure}
%\includegraphics[height=1in, width=1in]{rosette}
%\caption{A sample black and white graphic that has
%been resized with the \texttt{includegraphics} command.}
%\end{figure}

%\end{document}  % This is where a 'short' article might terminate



\appendix
%Appendix A
\section{Code listing}
% This next section command marks the start of
% Appendix B, and does not continue the present hierarchy
\section{Anything else?}

\begin{acks}
  The authors would like to thank the CISL and RAL Visitor Program

  The work is
  supported by the \grantsponsor{GSxxxxx}{National
  Science Foundation China}{http://dx.doi.org/zz.yyyyy/xxxxx} under Grant
  No.:~\grantnum{GSxxxxx}{yyyyyyy}

\end{acks}


\bibliographystyle{ACM-Reference-Format}
\bibliography{bibliography}

\end{document}
